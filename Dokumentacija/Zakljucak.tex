\chapter{Zaključak i budući rad}
		
		%\textbf{\textit{dio 2. revizije}}\\
		
		 %\textit{U ovom poglavlju potrebno je napisati osvrt na vrijeme izrade projektnog zadatka, koji su tehnički izazovi prepoznati, jesu li riješeni ili kako bi mogli biti riješeni, koja su znanja stečena pri izradi projekta, koja bi znanja bila posebno potrebna za brže i kvalitetnije ostvarenje projekta i koje bi bile perspektive za nastavak rada u projektnoj grupi.}
		
		 %\textit{Potrebno je točno popisati funkcionalnosti koje nisu implementirane u ostvarenoj aplikaciji.}
		
		\section{Analiza zadatka}
		\par{
		    Timski zadatak bio je analizirati traženi sustav, podijeliti se na podtimove koji će se baviti pojedinim aspektima sustava i naposljetku napraviti dogovorene zadatke.
		}
		\section{Postignuti rezultati}
		\par{
		    Unatoč ne tako dugom vremenskom roku i vrlo ograničenom predznanju svih članova tima, iznimno smo zadovoljni postignutim rezultatom te smatramo da smo zadatak ispunili u potpunosti. Zadatke smo izvršavali redovito, a nismo se ustručavali ni tražiti pomoć od asistenta kada bismo negdje zapeli. Sastanke s asistentom koristili smo kao sastanke s naručiteljem projekta, na kojima bismo razjašnjavali nedoumice oko zadatka, tražili dodatne informacije i potpitanjima sortirali prioritete pri implementaciji.
		}
		\par{
		    Zadatak smo rješavali strukturirano, najprije razmatrajući obrasce uporabe sustava, a potom smo krenuli na konkretnu implementaciju. Tim smo podijelili u dio koji se bavio poslužiteljskom stranom (\textit{backend}), dio koji se bavio klijentskom stranom (\textit{frontend}) i dio koji se bavio izradom dokumentacije. Timovi nisu bili statični, već je u vremenu dolazilo do izmjena, kako zbog potrebe, tako i zbog želje članova tima da iskoriste rad na projektu kao priliku za učenje novih tehnologija koje će na budućim projektima moći koristiti kao korisno predznanje.
		}
		\section{Izazovi u radu}
		\par{
		    Izazovi na koje smo nailazili najčešće su bili vezani uz nedostatak predznanja, a najčešće smo ih rješavali ili samostalno tražeći odgovore na internetu, ili u komunikaciji s ostalim članovima tima koji su nailazili na slične probleme. Kao što je već spomenuto, probleme koje nismo znali riješiti sami proslijedili smo asistentima kako bismo što prije mogli nastaviti dalje sa implementacijom. Osim toga, ponekad je problem znao predstavljati nedostatak komunikacije, točnije stajanje određenih dijelova projekta zbog neadekvatnog prenošenja informacija. Ipak, do kraja projekta naučili smo iz pogrešaka i poboljšali aspekte koje smo mogli.
		}
		\par{
		    Izazov koji nas je pratio kroz cijelo trajanje projekta je nedostatak vremena, što zbog relativno kratkog vremenskog roka, to i zbog ostalih fakultetskih i poslovnih obaveza. Predmet \textit{Programsko inženjerstvo} definira opterećenje nesrazmjerno stvarnom opterećenju, pogotovo s obzirom na udio bodova koji nosi izrada ovog projekta (što se izravno može vidjeti iz tablice sa satima utrošenima na projekt).
		}
		\section{Mogućnosti za daljnji rad}
		\par{
		    Ipak, bez obzira na nesrazmjerno nastavno opterećenje, ovaj projekt jedna je od najkorisnijih stvari koje smo dosad radili na fakultetu, pružajući nam uvid u rad u timu na konkretnom projektnom zadatku, a to je posao s kojim ćemo se svi u timu jednog dana vrlo vjerojatno susresti. Ne samo da smo stekli puno konkretnog znanja, nego smo se poboljšali i u međuljudskim odnosima i bolje pripremili za buduće radno mjesto.
		}
		\par{
		    Kako je opisano u poglavlju sa opisom projektnog zadatka, mnogo nam je dodatnih ideja padalo na pamet. Te ideje uglavnom nismo implementirali, ali su ostavljene kao mogućnost za budući rad. Iako se ovaj projekt koristi primarno za nastavu, daljnji rad i implementacija dodatno osmišljenih funkcionalnosti bio bi odlična vježba za daljnje usavršavanje članova tima. S obzirom na ugodnu radnu atmosferu i pokazanu odgovornost članova tima, ostaje otvorena mogućnost za daljnji rad na projektu.
		}
		
		\eject 